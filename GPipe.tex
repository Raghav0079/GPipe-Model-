\documentclass[12pt]{article}

\usepackage{geometry}
\usepackage{hyperref}
\usepackage{graphicx}
\usepackage{enumitem}  % Added for proper enumeration
\setlist[enumerate]{leftmargin=1cm} % Set left margin for enumerate globally
\usepackage{float}     % Added for precise table placement
\usepackage{multirow}  % Added for multirow in tables
\usepackage{tikz}      % Added for tikzpicture diagrams
\usetikzlibrary{positioning} % Added for node positioning (right=of, above=of, etc.)
\date{} % This disables the date in the output PDF
\title{GPipe Analysis}
\author{Raghav Mishra}


\begin{document}

\maketitle

\section{First Pass}
\begin{itemize}
  \item It is used mainly for efficient and task-independent model parallelism.
  \item GPipe, a pipeline parallelism library that allows scaling any network that can be expressed as a sequence of layers.
  \item Provides the flexibility of scaling a variety of different networks to gigantic sizes efficiently.
  \item Training large-scale neural networks on two different tasks:
    \begin{enumerate}[label=(\roman*)]
      \item Image Classification: Training a 557-million-parameter AmoebaNet model that attains a top-1 accuracy of 84.4\% on ImageNet-2012.
      \item Multilingual Neural Machine Translation: Training a single 6-billion-parameter, 128-layer Transformer model on a corpus spanning over 100 languages, achieving better quality than all bilingual models.
    \end{enumerate}
\end{itemize}

\section{Second Pass}
\begin{itemize}
  \item GPipe allows scaling arbitrary deep neural network architectures beyond the memory limitations of a single accelerator by partitioning the model across different accelerators and supporting re-materialization on every accelerator.
  \item Gradient updates using GPipe are consistent regardless of the number of partitions, allowing researchers to easily train increasingly large models by deploying more accelerators.
  \item Experimental results on two tasks:
    \begin{enumerate}[label=(\roman*)]
      \item Image classification: Training the AmoebaNet model on 480 × 480 input from the ImageNet 2012 dataset. By increasing the model width, they scale up the number of parameters to 557 million and achieve a top-1 validation accuracy of 84.4\%.
      \item Machine translation: Training a single 128-layer 6-billion-parameter multilingual Transformer model on 103 languages (102 languages to English).
    \end{enumerate}
  \item Model is capable of outperforming the individually trained 350-million-parameter bilingual Transformer Big models on 100 language pairs.
  \item Introduction of GPipe : a scalable model-parallelism library for training giant neural networks
  \item Novel batch-splitting pipeline-parallelism algorithm that uses synchronous gradient updates, allowing model parallelism with high hardware utilization and training stability.
  \item GPipe performance with two very different types of model architectures: an AmoebaNet  convolutional model and a Transformer  sequence-to-sequence model.
  \item Study of scalability , efficiency and communication cost 
  \item Amoeba Net : experiments on Cloud TPUv2s 8GB memory accelerator
  \item GPipe reduces the intermediate activation memory requirements from 6.26GB to 3.46GB.
  \item Enables a 318M parameter model on a single accelerator
  \item Trained Transformer model on Cloud TPUv3s with 16GB memory per accelerator core. Fixed vocab size of 32k, seq length of 1024 and batch size of 32.
  \item Transformer layer has 2048 for model dimension, 8192 for feed-forward hidden dimension and 32 attention heads
  \item Re-materialization allows training a 2.7× larger model on a single accelerator. With 128 partitions, GPipe allows scaling Transformer up to 83.9B parameters, a 298× increase.
  \item the number of micro-batches M is at least 4× the number of partitions, the bubble overhead is almost negligible
  \item Transformer model, there is a 3.5× speedup when it is partitioned across four times more accelerators
  \begin{table}[H]
\centering
\caption{Normalized training throughput using GPipe with different number of partitions ($K$) and micro-batches ($M$) on TPUs. Performance increases with more micro-batches. There is an almost linear speedup with the number of accelerators for Transformer model when $M \gg K$. Batch size was adjusted to fit memory if necessary.}
\begin{tabular}{|c|c|c|c|c|c|c|}
\hline
\multirow{2}{*}{$M$} & \multicolumn{3}{c|}{AmoebaNet} & \multicolumn{3}{c|}{Transformer} \\
\cline{2-7}
 & $K=2$ & $K=4$ & $K=8$ & $K=2$ & $K=4$ & $K=8$ \\
\hline
1  & 1    & 1.13 & 1.38 & 1   & 1.07 & 1.3 \\
4  & 1.07 & 1.26 & 1.72 & 1.7 & 3.2  & 4.8 \\
32 & 1.21 & 1.84 & 3.48 & 1.8 & 3.4  & 6.3 \\
\hline
\end{tabular}
\end{table}
\begin{table}[H]
\centering
\caption{Normalized training throughput using GPipe on GPUs without high-speed interconnect.}
\begin{tabular}{|c|c|c|c|c|c|c|}
\hline
\multirow{2}{*}{$M$} & \multicolumn{3}{c|}{AmoebaNet} & \multicolumn{3}{c|}{Transformer} \\
\cline{2-7}
 & $K=2$ & $K=4$ & $K=8$ & $K=2$ & $K=4$ & $K=8$ \\
\hline
32 & 1 & 1.7 & 2.7 & 1 & 1.8 & 3.3 \\
\hline
\end{tabular}
\end{table}
\begin{table}[H]
\centering
\caption{Image classification accuracy using AmoebaNet-B (18, 512) first trained on ImageNet 2012 then fine-tuned on others. Fine-tuned results averaged across 5 runs. Baseline results from Real et al. [12] and Cubuk et al. [26] were directly trained from scratch. *Mahajan et al.’s model [27] achieved 85.4\% top-1 accuracy but was pretrained on non-public Instagram data. Ngiam et al. [28] achieved better results by pre-training with data from a private dataset (JFT-300M).}
\begin{tabular}{|l|r|r|r|c|c|}
\hline
Dataset         & \# Train  & \# Test  & \# Classes & Accuracy (\%) & Previous Best (\%) \\
\hline
ImageNet-2012   & 1,281,167 & 50,000   & 1000       & 84.4          & 83.9 [12] (85.4$^*$[27]) \\
CIFAR-10        & 50,000    & 10,000   & 10         & 99.0          & 98.5 [26] \\
CIFAR-100       & 50,000    & 10,000   & 100        & 91.3          & 89.3 [26] \\
Stanford Cars   & 8,144     & 8,041    & 196        & 94.6          & 94.8$^*$ [26] \\
Oxford Pets     & 3,680     & 3,369    & 37         & 95.9          & 93.8$^*$ [29] \\
Food-101        & 75,750    & 25,250   & 101        & 93.0          & 90.4$^*$ [30] \\
FGVC Aircraft   & 6,667     & 3,333    & 100        & 92.7          & 92.9$^*$ [31] \\
Birdsnap        & 47,386    & 2,443    & 500        & 83.6          & 80.2$^*$ [32] \\
\hline
\end{tabular}
\end{table}
\end{itemize}
\section{Third Pass}
\begin{enumerate}
  \item GPipe enables large-scale model parallelism using pipeline parallelism with micro-batches.  
  \item Models are divided into sequential partitions that run on multiple accelerators.  
  \item Micro-batch scheduling ensures high utilization and reduces idle time.  
  \item It eliminates the memory constraints of single-device training.  
  \item GPipe supports scaling models both in width and depth.  
  \item The approach reduces communication overhead compared to naive model parallelism.  
  \item GPipe works efficiently with synchronous training.  
  \item Pipeline parallelism overlaps computation and communication.  
  \item Checkpoints allow efficient memory use via re-materialization.  
  \item GPipe scales to very large models with minimal code modifications.  
  \item Experiments with CIFAR-10 validated the effectiveness of GPipe.  
  \item AmoebaNet was used as a benchmark model for image classification.  
  \item Scaling depth and width improved test accuracy significantly.  
  \item GPipe achieved near-linear speedup with more devices.  
  \item Partitioning into 4 accelerators improved throughput compared to 2.  
  \item Even with communication overhead, training remained efficient.  
  \item AmoebaNet models trained with GPipe reached state-of-the-art accuracy on CIFAR-10.  
  \item Training larger networks became feasible without GPU OOM errors.  
  \item Memory savings allowed deeper models to fit within hardware limits.  
  \item GPipe provides flexibility for both research prototypes and production systems.  
  \item GPipe was further applied to large-scale NLP, specifically multilingual machine translation (MMT).  
  \item The dataset included 25 billion parallel sentences across 102 languages and English.  
  \item Data spanned both low-resource and high-resource languages.  
  \item Transformer models were scaled along depth (layers) and width (hidden size, attention heads).  
  \item The base Transformer Big model had 400M parameters, denoted as $T(6,8192,16)$.  
  \item Scaling produced models of 1.3B, 3B, and 6B parameters.  
  \item Increasing model capacity improved BLEU scores across all languages.  
  \item The 1.3B deep model outperformed the wide model for low-resource languages.  
  \item Depth provided stronger generalization and transfer learning benefits.  
  \item Larger models yielded diminishing returns beyond 3B parameters.  
  \item Training instability was observed with deeper models due to sharp activations and noise.  
  \item Logit clipping and scaled initialization were applied to stabilize training.  
  \item These methods reduced exploding gradients and preserved convergence.  
  \item Temperature-based sampling improved multilingual performance.  
  \item Larger models significantly improved translation for low-resource languages.  
  \item Bilingual baselines were outperformed by multilingual GPipe models.  
  \item Model parallelism allowed training across up to 16 accelerators.  
  \item BLEU scores increased consistently with model size.  
  \item Data parallelism with very large batch sizes was also explored.  
  \item Batch sizes scaled from 260K tokens to 4M tokens per step.  
  \item The largest batch ever reported in NMT (4M tokens) improved BLEU performance.  
  \item Increasing batch size reduced validation loss significantly.  
  \item Both low-resource and high-resource language pairs benefited.  
  \item GPipe’s efficiency enabled training such massive batches.  
  \item Depth scaling was particularly beneficial for scarce languages.  
  \item Width scaling primarily helped high-resource languages.  
  \item A trade-off exists: deeper models generalize better, wider models converge faster.  
  \item Pipeline parallelism enables distributed training with near-linear scaling.  
  \item The combination of GPipe and Transformer scaling demonstrated new state-of-the-art in MMT.  
  \item GPipe thus provides a general-purpose solution for scaling deep learning across domains.  


% ---------------- GPipe Core ------------------
\item GPipe enables large-scale model parallelism using pipeline parallelism with micro-batches.  
\item Models are divided into sequential partitions that run on multiple accelerators.  
\item Micro-batch scheduling ensures high utilization and reduces idle time.  
\item It eliminates the memory constraints of single-device training.  
\item GPipe supports scaling models both in width and depth.  
\item The approach reduces communication overhead compared to naive model parallelism.  
\item GPipe works efficiently with synchronous training.  
\item Pipeline parallelism overlaps computation and communication.  
\item Checkpoints allow efficient memory use via re-materialization.  
\item GPipe scales to very large models with minimal code modifications.  


% ---------------- CIFAR-10 & AmoebaNet ------------------
\item Experiments with CIFAR-10 validated the effectiveness of GPipe.  
\item AmoebaNet was used as a benchmark model for image classification.  
\item Scaling depth and width improved test accuracy significantly.  
\item GPipe achieved near-linear speedup with more devices.  
\item Partitioning into 4 accelerators improved throughput compared to 2.  
\item Even with communication overhead, training remained efficient.  
\item AmoebaNet models trained with GPipe reached state-of-the-art accuracy on CIFAR-10.  
\item Training larger networks became feasible without GPU OOM errors.  
\item Memory savings allowed deeper models to fit within hardware limits.  
\item GPipe provides flexibility for both research prototypes and production systems.  

% ---------------- Multilingual Machine Translation ------------------
\item GPipe was further applied to large-scale NLP, specifically multilingual machine translation (MMT).  
\item The dataset included 25 billion parallel sentences across 102 languages and English.  
\item Data spanned both low-resource and high-resource languages.  
\item Transformer models were scaled along depth (layers) and width (hidden size, attention heads).  
\item The base Transformer Big model had 400M parameters, denoted as $T(6,8192,16)$.  
\item Scaling produced models of 1.3B, 3B, and 6B parameters.  
\item Increasing model capacity improved BLEU scores across all languages.  
\item The 1.3B deep model outperformed the wide model for low-resource languages.  
\item Depth provided stronger generalization and transfer learning benefits.  
\item Larger models yielded diminishing returns beyond 3B parameters.  

% ---------------- Empirical Findings ------------------
\item Training instability was observed with deeper models due to sharp activations and noise.  
\item Logit clipping and scaled initialization were applied to stabilize training.  
\item These methods reduced exploding gradients and preserved convergence.  
\item Temperature-based sampling improved multilingual performance.  
\item Larger models significantly improved translation for low-resource languages.  
\item Bilingual baselines were outperformed by multilingual GPipe models.  
\item Model parallelism allowed training across up to 16 accelerators.  
\item BLEU scores increased consistently with model size.  
\item Data parallelism with very large batch sizes was also explored.  
\item Batch sizes scaled from 260K tokens to 4M tokens per step.  

% ---------------- Large-Batch Findings ------------------
\item The largest batch ever reported in NMT (4M tokens) improved BLEU performance.  
\item Increasing batch size reduced validation loss significantly.  
\item Both low-resource and high-resource language pairs benefited.  
\item GPipe’s efficiency enabled training such massive batches.  
\item Depth scaling was particularly beneficial for scarce languages.  
\item Width scaling primarily helped high-resource languages.  
\item A trade-off exists: deeper models generalize better, wider models converge faster.  
\item Pipeline parallelism enables distributed training with near-linear scaling.  
\item The combination of GPipe and Transformer scaling demonstrated new state-of-the-art in MMT.  
\item GPipe thus provides a general-purpose solution for scaling deep learning across domains.  

\end{enumerate}

\section{GPipe Parallelism Diagrams}
\resizebox{\textwidth}{!}{
\begin{tikzpicture}[
    node distance=1cm and 1cm,
    box/.style={draw, thick, minimum width=1.2cm, minimum height=0.7cm, align=center},
    fwd/.style={box, fill=blue!20},
    bwd/.style={box, fill=brown!20},
    loss/.style={box, fill=red!20},
    update/.style={box, fill=green!20}
]

% ------------------- (a) Model Parallelism -------------------
\node[fwd] (f0) {F$_0$};
\node[bwd, right=of f0] (b0) {B$_0$};

\node[fwd, above=of f0] (f1) {F$_1$};
\node[bwd, right=of f1] (b1) {B$_1$};

\node[fwd, above=of f1] (f2) {F$_2$};
\node[bwd, right=of f2] (b2) {B$_2$};

\node[fwd, above=of f2] (f3) {F$_3$};
\node[bwd, right=of f3] (b3) {B$_3$};

\node[loss, above=0.8cm of f3, minimum width=3.5cm] (loss) {Loss};

% connections
\draw[->, thick] (f0) -- (b0);
\draw[->, thick] (f1) -- (b1);
\draw[->, thick] (f2) -- (b2);
\draw[->, thick] (f3) -- (b3);

\draw[->, thick] (f0) -- (f1);
\draw[->, thick] (f1) -- (f2);
\draw[->, thick] (f2) -- (f3);

\draw[->, thick] (b3) -- (b2);
\draw[->, thick] (b2) -- (b1);
\draw[->, thick] (b1) -- (b0);

\draw[->, thick] (f3) -- (loss);
\draw[->, thick] (b3) -- (loss);

\node[below=0.8cm of f0, minimum width=3.5cm, align=center] (grad) {Gradients};

\draw[->, thick] (b0) -- (grad);

\node[below left=0.2cm and -2cm of f0, align=left] {(a)};

% ------------------- (b) Naive Pipeline -------------------
\node[fwd, right=8cm of f0] (nf0) {F$_0$};
\node[bwd, right=of nf0] (nb0) {B$_0$};
\node[fwd, below=of nf0] (nf1) {F$_0$};
\node[bwd, right=of nf1] (nb1) {B$_0$};

\draw[->, thick] (nf0) -- (nb0);
\draw[->, thick] (nf1) -- (nb1);

\node[right=of nb0, align=center] {Update};

\node[below left=0.2cm and -1.2cm of nf1, align=left] {(b)};

% ------------------- (c) Pipeline Parallelism -------------------
\node[fwd, below=6cm of nf1] (cf00) {F$_{0,0}$};
\node[fwd, right=of cf00] (cf01) {F$_{0,1}$};
\node[fwd, right=of cf01] (cf02) {F$_{0,2}$};

\node[fwd, above=of cf01] (cf10) {F$_{1,0}$};
\node[fwd, right=of cf10] (cf11) {F$_{1,1}$};
\node[fwd, right=of cf11] (cf12) {F$_{1,2}$};

\node[fwd, above=of cf11] (cf20) {F$_{2,0}$};
\node[fwd, right=of cf20] (cf21) {F$_{2,1}$};

\node[fwd, above=of cf21] (cf30) {F$_{3,0}$};
\node[fwd, right=of cf30] (cf31) {F$_{3,1}$};

\node at ($(cf11)!0.5!(cf21)+(3,0)$) {Bubble};

\node[below left=0.2cm and -2.5cm of cf00, align=left] {(c)};
\end{tikzpicture}
}
\end{tikzpicture}


\end{document}